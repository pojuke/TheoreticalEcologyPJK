% Options for packages loaded elsewhere
\PassOptionsToPackage{unicode}{hyperref}
\PassOptionsToPackage{hyphens}{url}
%
\documentclass[
]{book}
\usepackage{amsmath,amssymb}
\usepackage{iftex}
\ifPDFTeX
  \usepackage[T1]{fontenc}
  \usepackage[utf8]{inputenc}
  \usepackage{textcomp} % provide euro and other symbols
\else % if luatex or xetex
  \usepackage{unicode-math} % this also loads fontspec
  \defaultfontfeatures{Scale=MatchLowercase}
  \defaultfontfeatures[\rmfamily]{Ligatures=TeX,Scale=1}
\fi
\usepackage{lmodern}
\ifPDFTeX\else
  % xetex/luatex font selection
\fi
% Use upquote if available, for straight quotes in verbatim environments
\IfFileExists{upquote.sty}{\usepackage{upquote}}{}
\IfFileExists{microtype.sty}{% use microtype if available
  \usepackage[]{microtype}
  \UseMicrotypeSet[protrusion]{basicmath} % disable protrusion for tt fonts
}{}
\makeatletter
\@ifundefined{KOMAClassName}{% if non-KOMA class
  \IfFileExists{parskip.sty}{%
    \usepackage{parskip}
  }{% else
    \setlength{\parindent}{0pt}
    \setlength{\parskip}{6pt plus 2pt minus 1pt}}
}{% if KOMA class
  \KOMAoptions{parskip=half}}
\makeatother
\usepackage{xcolor}
\usepackage{color}
\usepackage{fancyvrb}
\newcommand{\VerbBar}{|}
\newcommand{\VERB}{\Verb[commandchars=\\\{\}]}
\DefineVerbatimEnvironment{Highlighting}{Verbatim}{commandchars=\\\{\}}
% Add ',fontsize=\small' for more characters per line
\usepackage{framed}
\definecolor{shadecolor}{RGB}{248,248,248}
\newenvironment{Shaded}{\begin{snugshade}}{\end{snugshade}}
\newcommand{\AlertTok}[1]{\textcolor[rgb]{0.94,0.16,0.16}{#1}}
\newcommand{\AnnotationTok}[1]{\textcolor[rgb]{0.56,0.35,0.01}{\textbf{\textit{#1}}}}
\newcommand{\AttributeTok}[1]{\textcolor[rgb]{0.13,0.29,0.53}{#1}}
\newcommand{\BaseNTok}[1]{\textcolor[rgb]{0.00,0.00,0.81}{#1}}
\newcommand{\BuiltInTok}[1]{#1}
\newcommand{\CharTok}[1]{\textcolor[rgb]{0.31,0.60,0.02}{#1}}
\newcommand{\CommentTok}[1]{\textcolor[rgb]{0.56,0.35,0.01}{\textit{#1}}}
\newcommand{\CommentVarTok}[1]{\textcolor[rgb]{0.56,0.35,0.01}{\textbf{\textit{#1}}}}
\newcommand{\ConstantTok}[1]{\textcolor[rgb]{0.56,0.35,0.01}{#1}}
\newcommand{\ControlFlowTok}[1]{\textcolor[rgb]{0.13,0.29,0.53}{\textbf{#1}}}
\newcommand{\DataTypeTok}[1]{\textcolor[rgb]{0.13,0.29,0.53}{#1}}
\newcommand{\DecValTok}[1]{\textcolor[rgb]{0.00,0.00,0.81}{#1}}
\newcommand{\DocumentationTok}[1]{\textcolor[rgb]{0.56,0.35,0.01}{\textbf{\textit{#1}}}}
\newcommand{\ErrorTok}[1]{\textcolor[rgb]{0.64,0.00,0.00}{\textbf{#1}}}
\newcommand{\ExtensionTok}[1]{#1}
\newcommand{\FloatTok}[1]{\textcolor[rgb]{0.00,0.00,0.81}{#1}}
\newcommand{\FunctionTok}[1]{\textcolor[rgb]{0.13,0.29,0.53}{\textbf{#1}}}
\newcommand{\ImportTok}[1]{#1}
\newcommand{\InformationTok}[1]{\textcolor[rgb]{0.56,0.35,0.01}{\textbf{\textit{#1}}}}
\newcommand{\KeywordTok}[1]{\textcolor[rgb]{0.13,0.29,0.53}{\textbf{#1}}}
\newcommand{\NormalTok}[1]{#1}
\newcommand{\OperatorTok}[1]{\textcolor[rgb]{0.81,0.36,0.00}{\textbf{#1}}}
\newcommand{\OtherTok}[1]{\textcolor[rgb]{0.56,0.35,0.01}{#1}}
\newcommand{\PreprocessorTok}[1]{\textcolor[rgb]{0.56,0.35,0.01}{\textit{#1}}}
\newcommand{\RegionMarkerTok}[1]{#1}
\newcommand{\SpecialCharTok}[1]{\textcolor[rgb]{0.81,0.36,0.00}{\textbf{#1}}}
\newcommand{\SpecialStringTok}[1]{\textcolor[rgb]{0.31,0.60,0.02}{#1}}
\newcommand{\StringTok}[1]{\textcolor[rgb]{0.31,0.60,0.02}{#1}}
\newcommand{\VariableTok}[1]{\textcolor[rgb]{0.00,0.00,0.00}{#1}}
\newcommand{\VerbatimStringTok}[1]{\textcolor[rgb]{0.31,0.60,0.02}{#1}}
\newcommand{\WarningTok}[1]{\textcolor[rgb]{0.56,0.35,0.01}{\textbf{\textit{#1}}}}
\usepackage{longtable,booktabs,array}
\usepackage{calc} % for calculating minipage widths
% Correct order of tables after \paragraph or \subparagraph
\usepackage{etoolbox}
\makeatletter
\patchcmd\longtable{\par}{\if@noskipsec\mbox{}\fi\par}{}{}
\makeatother
% Allow footnotes in longtable head/foot
\IfFileExists{footnotehyper.sty}{\usepackage{footnotehyper}}{\usepackage{footnote}}
\makesavenoteenv{longtable}
\usepackage{graphicx}
\makeatletter
\def\maxwidth{\ifdim\Gin@nat@width>\linewidth\linewidth\else\Gin@nat@width\fi}
\def\maxheight{\ifdim\Gin@nat@height>\textheight\textheight\else\Gin@nat@height\fi}
\makeatother
% Scale images if necessary, so that they will not overflow the page
% margins by default, and it is still possible to overwrite the defaults
% using explicit options in \includegraphics[width, height, ...]{}
\setkeys{Gin}{width=\maxwidth,height=\maxheight,keepaspectratio}
% Set default figure placement to htbp
\makeatletter
\def\fps@figure{htbp}
\makeatother
\setlength{\emergencystretch}{3em} % prevent overfull lines
\providecommand{\tightlist}{%
  \setlength{\itemsep}{0pt}\setlength{\parskip}{0pt}}
\setcounter{secnumdepth}{5}
\usepackage{booktabs}
\usepackage{amsthm}
\makeatletter
\def\thm@space@setup{%
  \thm@preskip=8pt plus 2pt minus 4pt
  \thm@postskip=\thm@preskip
}
\makeatother
\usepackage{booktabs}
\usepackage{longtable}
\usepackage{array}
\usepackage{multirow}
\usepackage{wrapfig}
\usepackage{float}
\usepackage{colortbl}
\usepackage{pdflscape}
\usepackage{tabu}
\usepackage{threeparttable}
\usepackage{threeparttablex}
\usepackage[normalem]{ulem}
\usepackage{makecell}
\usepackage{xcolor}
\ifLuaTeX
  \usepackage{selnolig}  % disable illegal ligatures
\fi
\usepackage[]{natbib}
\bibliographystyle{apalike}
\usepackage{bookmark}
\IfFileExists{xurl.sty}{\usepackage{xurl}}{} % add URL line breaks if available
\urlstyle{same}
\hypersetup{
  pdftitle={Introduction to Theoretical Ecology},
  pdfauthor={Instructor: Po-Ju Ke \textasciitilde\textasciitilde\textasciitilde\textasciitilde\textasciitilde{} Teaching Assistant: Hsiang-Chih Lo},
  hidelinks,
  pdfcreator={LaTeX via pandoc}}

\title{Introduction to Theoretical Ecology}
\author{Instructor: Po-Ju Ke \(~~~~~\) Teaching Assistant: Hsiang-Chih Lo}
\date{2024 Fall at National Taiwan Univeristy \includegraphics{./bifurcation.gif}}

\begin{document}
\maketitle

{
\setcounter{tocdepth}{1}
\tableofcontents
}
\chapter*{Course information}\label{course-information}
\addcontentsline{toc}{chapter}{Course information}

\textbf{Description}

The development of theory plays an important role in advancing ecology as a scientific field. This three-unit course is for students at the graduate or advanced undergraduate level. The course will cover classic theoretical topics in population and community ecology, staring from single-species dynamics and gradually build up to multispecies models. Emphasis will be on theoretical concepts and corresponding mathematical approaches.

This course is designed as a two-hour lecture (written on black board) followed by a one-hour complementary hands-on practice module. In the lecture, we will analyze dynamical models and discuss their theoretical implications. In the practice section, we will use a combination interactive applications and numerical simulations to gain more intuition of the dynamics and behavior of different models.

\textbf{Objective}

By the end of the course, students are expected to be familiar with the basic building blocks of ecological models, and would be able to formulate and analyze simple models of their own. The hands-on practice component should allow students to link their ecological intuition with the underlying mathematical model, helping them to better understand the primary literature of theoretical ecology.

\textbf{Requirement}

Students are only expected to have a basic understanding of \textbf{Calculus} (e.g., freshman introductory course) and \textbf{Ecology}. It's OK if you're not familiar with calculus as we will provide relevant material for you to review during the first week.

\textbf{Format}

Tuesday 6,7,8 (1:20 pm \textasciitilde{} 4:20 pm) at 共207

\textbf{Grading}

The final grade consists of:

\begin{enumerate}
\def\labelenumi{(\arabic{enumi})}
\tightlist
\item
  Assignment problem sets (60\%)
\item
  Midterm exam (15\%)
\item
  Final exam (15\%)
\item
  Course participation (10\%)
\end{enumerate}

\textbf{Course materials}

We will use a combination of textbooks of theoretical ecology. Textbook chapters and additional reading materials (listed in the course outline) will be provided. (see \href{https://pojuke.github.io/TheoreticalEcologyPJK/syllabus.html}{\textbf{Syllabus}} for more details).

Below are the textbook references:

\begin{enumerate}
\def\labelenumi{(\arabic{enumi})}
\tightlist
\item
  \emph{A Primer of Ecology} (4\textsuperscript{th} edition). Nicholas Gotelli, 2008.
\item
  \emph{An Illustrated Guide to Theoretical Ecology}. Ted Case, 2000.
\item
  \emph{A Biologist's Guide to Mathematical Modeling in Ecology and Evolution}. Sarah Otto \& Troy Day, 2011.
\item
  \emph{Mathematical Ecology of Populations and Ecosystems}. John Pastor, 2008.
\item
  \emph{Nonlinear Dynamics and Choas}. Steven Strogatz, 2000.
\end{enumerate}

\textbf{Contacts}

\textbf{Instructor}: Po-Ju Ke

\begin{itemize}
\tightlist
\item
  Office: Life Science Building R635
\item
  Email: \href{mailto:pojuke@ntu.edu.tw}{\nolinkurl{pojuke@ntu.edu.tw}}
\item
  Office hours: by appointment
\end{itemize}

\textbf{Teaching assistant}: Hsiang-Chih Lo

\begin{itemize}
\tightlist
\item
  Office: Life Science Building R635
\item
  Email: \href{mailto:r12b44009@ntu.edu.tw}{\nolinkurl{r12b44009@ntu.edu.tw}}
\item
  Office hours:
\end{itemize}

\chapter*{Syllabus}\label{syllabus}
\addcontentsline{toc}{chapter}{Syllabus}

\begingroup\fontsize{17}{19}\selectfont

\begin{tabu} to \linewidth {>{\centering}X>{\centering}X>{\centering}X>{\raggedright}X}
\hline
\begingroup\fontsize{20}{22}\selectfont \textcolor{black}{\textbf{Date}}\endgroup & \begingroup\fontsize{20}{22}\selectfont \textcolor{black}{\textbf{Lecture topic}}\endgroup & \begingroup\fontsize{20}{22}\selectfont \textcolor{black}{\textbf{Lab}}\endgroup & \begingroup\fontsize{20}{22}\selectfont \textcolor{black}{\textbf{Readings}}\endgroup\\
\hline
**Week 1** <span style='vertical-align:-30%'> </span>
           <br> 9/3 & Introduction: what is theoretical ecology? & \- & [**Grainger et al., 2021**](https://doi.org/10.1086/717206)\\
\hline
**Week 2** <span style='vertical-align:-30%'> </span>
           <br> 9/10 & Exponential population growth & Solving exponential growth equation using "deSolve" & Visualization & Gotelli [Ch.1], Case[Ch.1]\\
\hline
**Week 3** <span style='vertical-align:-30%'> </span>
           <br> 9/17 & Logistic population growth and stability analysis & Shinny App for logistic population growth & Gotelli [Ch.2], Case[Ch.5], Otto & Day[Ch.5]\\
\hline
**Week 4** <span style='vertical-align:-30%'> </span>
           <br> 9/24 & Metapopulations and patch occupancy models & Alternative stable state diagram & Gotelli [Ch.4], Case[Ch.16]\\
\hline
**Week 5** <span style='vertical-align:-30%'> </span>
           <br> 10/1 & Harvesting and bifurcation & Metapopulations and patch occupancy models & Pastor [Ch. 7], Strogatz [Ch. 3]\\
\hline
**Week 6** <span style='vertical-align:-30%'> </span>
           <br> 10/8 & No class (National holiday) & \- & \-\\
\hline
**Week 7** <span style='vertical-align:-30%'> </span>
           <br> 10/15 & Age-structured population models & Age-structured population model & Gotelli [Ch.3], Case[Ch.3]\\
\hline
**Week 8** <span style='vertical-align:-30%'> </span>
           <br> 10/22 & Lotka-Volterra model of competition: graphical analysis & Lotka-Volterra competition model - Population dynamics & Gotelli [Ch.5], Case[Ch.14]\\
\hline
**Week 9** <span style='vertical-align:-30%'> </span>
           <br> 10/29 & **Midterm exam** & \- & \-\\
\hline
**Week 10** <span style='vertical-align:-30%'> </span>
           <br> 11/5 & Lotka-Volterra model of competition: invasion analysis and linear stability analysis & Lotka-Volterra competition model - Visualization of dynamics with complex eigenvalues & Otto & Day [Ch.8], 
               [**Broekman et al., 2019**]( https://doi.org/10.1111/ele.13349)\\
\hline
**Week 11** <span style='vertical-align:-30%'> </span>
           <br> 11/12 & Predator-prey interactions (I) & Lotka-Volterra model of predator-prey interactions and time-scale separation & Gotelli [Ch.6], Case[Ch.12, 13]\\
\hline
**Week 12** <span style='vertical-align:-30%'> </span>
           <br> 11/19 & Predator-prey interactions (II) + Discussion on May (1972) & Rosenzweig-MacArthur predator-prey model and May's complexity-stability relationship & Gotelli [Ch.6], Case[Ch.12, 13], 
               [**May., 1972**](https://www.nature.com/articles/238413a0)\\
\hline
**Week 13** <span style='vertical-align:-30%'> </span>
           <br> 11/26 & Mechanistic models for consumer-resource dynamics & Parameter space for apparent competition model & [**Tilman., 1980**](https://doi.org/10.1086/283633),
               [**Armstrong & McGehee., 1980**](https://doi.org/10.1086/283553)\\
\hline
**Week 14** <span style='vertical-align:-30%'> </span>
           <br> 12/3 & Multispecies models of predation: apparent competition & Resource competition & [**Holt., 1977**](https://doi.org/10.1016/0040-5809(77)90042-9)\\
\hline
**Week 15** <span style='vertical-align:-30%'> </span>
           <br> 12/10 & Research applcations: plant-soil feedback as an example & \- & \-\\
\hline
**Week 16** <span style='vertical-align:-30%'> </span>
           <br> 12/17 & **Final exam** & \- & \-\\
\hline
\end{tabu}
\endgroup{}

\chapter*{Week 2 - Exponential population growth}\label{week-2---exponential-population-growth}
\addcontentsline{toc}{chapter}{Week 2 - Exponential population growth}

In part 1, we will solve the differential equation for exponential population growth and visualize how the population sizes change over time.

\textbf{Part 1 - Numerical solution using the package \texttt{deSolve}}

Two main phases:

\begin{enumerate}
\def\labelenumi{(\arabic{enumi})}
\item
  Model specification: specify the structure of differential equation model
\item
  Model application: set the time steps, initial population size and model parameters (e.g., intrinsic population growth rate \(r\)), and then solve the equation model
\end{enumerate}

Consider the model
\[
\frac{dN}{dt} = rN
\]
where \(N\) is population size and \(r\) is the intrinsic growth rate.

\begin{Shaded}
\begin{Highlighting}[]
\DocumentationTok{\#\#\#\#\#\# part 1 \#\#\#\#\#\#}
\CommentTok{\# install.packages("deSolve")}
\FunctionTok{library}\NormalTok{(deSolve)}

\DocumentationTok{\#\#\# (1) Model specification}
\NormalTok{exponential\_model }\OtherTok{\textless{}{-}} \ControlFlowTok{function}\NormalTok{(times, state, parms) \{}
  \FunctionTok{with}\NormalTok{(}\FunctionTok{as.list}\NormalTok{(}\FunctionTok{c}\NormalTok{(state, parms)), \{}
\NormalTok{    dN\_dt }\OtherTok{=}\NormalTok{ r}\SpecialCharTok{*}\NormalTok{N  }\CommentTok{\# Exponential growth equation}
    \FunctionTok{return}\NormalTok{(}\FunctionTok{list}\NormalTok{(}\FunctionTok{c}\NormalTok{(dN\_dt)))  }\CommentTok{\# Return the results}
\NormalTok{  \})}
\NormalTok{\}}
\end{Highlighting}
\end{Shaded}

Set the time steps, initial population size and model parameters.

\begin{Shaded}
\begin{Highlighting}[]
\DocumentationTok{\#\#\# (2) Model application}
\NormalTok{times }\OtherTok{\textless{}{-}} \FunctionTok{seq}\NormalTok{(}\DecValTok{0}\NormalTok{, }\DecValTok{10}\NormalTok{, }\AttributeTok{by =} \FloatTok{0.1}\NormalTok{)  }\CommentTok{\# Time steps to integrate over}
\NormalTok{state }\OtherTok{\textless{}{-}} \FunctionTok{c}\NormalTok{(}\AttributeTok{N =} \DecValTok{10}\NormalTok{)  }\CommentTok{\# Initial population size}
\NormalTok{parms }\OtherTok{\textless{}{-}} \FunctionTok{c}\NormalTok{(}\AttributeTok{r =} \FloatTok{1.5}\NormalTok{)  }\CommentTok{\# Intrinsic growth rate}
\end{Highlighting}
\end{Shaded}

Solve the equation by \texttt{ode()} numerically.

\begin{Shaded}
\begin{Highlighting}[]
\CommentTok{\# Run the ode solver}
\NormalTok{pop\_size }\OtherTok{\textless{}{-}} \FunctionTok{ode}\NormalTok{(}\AttributeTok{func =}\NormalTok{ exponential\_model, }\AttributeTok{times =}\NormalTok{ times, }\AttributeTok{y =}\NormalTok{ state, }\AttributeTok{parms =}\NormalTok{ parms)}

\CommentTok{\# Take a look at the results}
\FunctionTok{head}\NormalTok{(pop\_size)}
\end{Highlighting}
\end{Shaded}

\begin{verbatim}
##      time        N
## [1,]  0.0 10.00000
## [2,]  0.1 11.61834
## [3,]  0.2 13.49860
## [4,]  0.3 15.68313
## [5,]  0.4 18.22120
## [6,]  0.5 21.17002
\end{verbatim}

Visualization

\begin{Shaded}
\begin{Highlighting}[]
\FunctionTok{plot}\NormalTok{(N }\SpecialCharTok{\textasciitilde{}}\NormalTok{ time, }\AttributeTok{data =}\NormalTok{ pop\_size)}
\end{Highlighting}
\end{Shaded}

\includegraphics{bookdown-demo_files/figure-latex/unnamed-chunk-5-1.pdf}

Compare simulation result with analytic solution, which is
\[
N(t) = N_0\exp\{rt\}
\]

\begin{Shaded}
\begin{Highlighting}[]
\FunctionTok{par}\NormalTok{(}\AttributeTok{mfrow =} \FunctionTok{c}\NormalTok{(}\DecValTok{1}\NormalTok{,}\DecValTok{2}\NormalTok{))}
\FunctionTok{plot}\NormalTok{(N }\SpecialCharTok{\textasciitilde{}}\NormalTok{ time, }\AttributeTok{data =}\NormalTok{ pop\_size) }\CommentTok{\# Plot simulation data}
\FunctionTok{curve}\NormalTok{(state[}\DecValTok{1}\NormalTok{]}\SpecialCharTok{*}\FunctionTok{exp}\NormalTok{(parms[}\DecValTok{1}\NormalTok{]}\SpecialCharTok{*}\NormalTok{x), }\AttributeTok{col =} \StringTok{"red"}\NormalTok{, }\AttributeTok{add =}\NormalTok{ T) }\CommentTok{\# Adding analytic solution}
\FunctionTok{plot}\NormalTok{(N }\SpecialCharTok{\textasciitilde{}}\NormalTok{ time, }\AttributeTok{data =}\NormalTok{ pop\_size, }\AttributeTok{log =} \StringTok{"y"}\NormalTok{) }\CommentTok{\# Plot logged simulation data}
\FunctionTok{curve}\NormalTok{(state[}\DecValTok{1}\NormalTok{]}\SpecialCharTok{*}\FunctionTok{exp}\NormalTok{(parms[}\DecValTok{1}\NormalTok{]}\SpecialCharTok{*}\NormalTok{x), }\AttributeTok{col =} \StringTok{"red"}\NormalTok{, }\AttributeTok{add =}\NormalTok{ T) }\CommentTok{\# Adding analytic solution}
\end{Highlighting}
\end{Shaded}

\includegraphics{bookdown-demo_files/figure-latex/unnamed-chunk-6-1.pdf}

\textbf{Part 2 - Comparing different ode solvers and different time intervals}
In default of \texttt{ode()}, the equations are solved by LSODA method. We can change the method by modifying the argument \texttt{method} in \texttt{ode()}.

\begin{Shaded}
\begin{Highlighting}[]
\DocumentationTok{\#\#\#\#\#\# part 2 \#\#\#\#\#\#}
\CommentTok{\# Original setting}
\NormalTok{times }\OtherTok{\textless{}{-}} \FunctionTok{seq}\NormalTok{(}\DecValTok{0}\NormalTok{, }\DecValTok{10}\NormalTok{, }\AttributeTok{by =} \FloatTok{0.1}\NormalTok{)  }\CommentTok{\# Time steps to integrate over}
\NormalTok{state }\OtherTok{\textless{}{-}} \FunctionTok{c}\NormalTok{(}\AttributeTok{N =} \DecValTok{10}\NormalTok{)  }\CommentTok{\# Initial population size}
\NormalTok{parms }\OtherTok{\textless{}{-}} \FunctionTok{c}\NormalTok{(}\AttributeTok{r =} \FloatTok{1.5}\NormalTok{)  }\CommentTok{\# Intrinsic growth rate}
\CommentTok{\# Default: LSODA}
\NormalTok{pop\_size }\OtherTok{\textless{}{-}} \FunctionTok{ode}\NormalTok{(}\AttributeTok{func =}\NormalTok{ exponential\_model, }\AttributeTok{times =}\NormalTok{ times, }\AttributeTok{y =}\NormalTok{ state, }\AttributeTok{parms =}\NormalTok{ parms)}

\CommentTok{\# Euler\textquotesingle{}s method}
\NormalTok{pop\_size\_1 }\OtherTok{\textless{}{-}} \FunctionTok{ode}\NormalTok{(}\AttributeTok{func =}\NormalTok{ exponential\_model, }\AttributeTok{times =}\NormalTok{ times, }\AttributeTok{y =}\NormalTok{ state, }\AttributeTok{parms =}\NormalTok{ parms, }\AttributeTok{method =} \StringTok{"euler"}\NormalTok{)}

\CommentTok{\# Compare different method}
\FunctionTok{par}\NormalTok{(}\AttributeTok{mfrow =} \FunctionTok{c}\NormalTok{(}\DecValTok{1}\NormalTok{,}\DecValTok{2}\NormalTok{))}
\FunctionTok{plot}\NormalTok{(N }\SpecialCharTok{\textasciitilde{}}\NormalTok{ time, }\AttributeTok{data =}\NormalTok{ pop\_size, }\AttributeTok{main =} \StringTok{"LSODA"}\NormalTok{)}
\FunctionTok{curve}\NormalTok{(state[}\DecValTok{1}\NormalTok{]}\SpecialCharTok{*}\FunctionTok{exp}\NormalTok{(parms[}\DecValTok{1}\NormalTok{]}\SpecialCharTok{*}\NormalTok{x), times[}\DecValTok{1}\NormalTok{], times[}\FunctionTok{length}\NormalTok{(times)], }\AttributeTok{col =} \StringTok{"red"}\NormalTok{, }\AttributeTok{add =}\NormalTok{ T) }\CommentTok{\# correct curve}
\FunctionTok{plot}\NormalTok{(N }\SpecialCharTok{\textasciitilde{}}\NormalTok{ time, }\AttributeTok{data =}\NormalTok{ pop\_size\_1, }\AttributeTok{main =} \StringTok{"Euler"}\NormalTok{)}
\FunctionTok{curve}\NormalTok{(state[}\DecValTok{1}\NormalTok{]}\SpecialCharTok{*}\FunctionTok{exp}\NormalTok{(parms[}\DecValTok{1}\NormalTok{]}\SpecialCharTok{*}\NormalTok{x), times[}\DecValTok{1}\NormalTok{], times[}\FunctionTok{length}\NormalTok{(times)], }\AttributeTok{col =} \StringTok{"red"}\NormalTok{, }\AttributeTok{add =}\NormalTok{ T) }\CommentTok{\# correct curve}
\end{Highlighting}
\end{Shaded}

\includegraphics{bookdown-demo_files/figure-latex/unnamed-chunk-7-1.pdf}

\begin{Shaded}
\begin{Highlighting}[]
\CommentTok{\# Different time intervals}
\NormalTok{times\_1 }\OtherTok{\textless{}{-}} \FunctionTok{seq}\NormalTok{(}\DecValTok{0}\NormalTok{, }\DecValTok{10}\NormalTok{, }\AttributeTok{by =} \DecValTok{1}\NormalTok{)  }\CommentTok{\# time steps to integrate over}
\NormalTok{times\_2 }\OtherTok{\textless{}{-}} \FunctionTok{seq}\NormalTok{(}\DecValTok{0}\NormalTok{, }\DecValTok{10}\NormalTok{, }\AttributeTok{by =} \FloatTok{0.1}\NormalTok{)  }\CommentTok{\# time steps to integrate over}
\NormalTok{times\_3 }\OtherTok{\textless{}{-}} \FunctionTok{seq}\NormalTok{(}\DecValTok{0}\NormalTok{, }\DecValTok{10}\NormalTok{, }\AttributeTok{by =} \FloatTok{0.01}\NormalTok{)  }\CommentTok{\# time steps to integrate over}

\CommentTok{\# Euler\textquotesingle{}s method}
\NormalTok{pop\_size\_1 }\OtherTok{\textless{}{-}} \FunctionTok{ode}\NormalTok{(}\AttributeTok{func =}\NormalTok{ exponential\_model, }\AttributeTok{times =}\NormalTok{ times\_1, }\AttributeTok{y =}\NormalTok{ state, }\AttributeTok{parms =}\NormalTok{ parms, }\AttributeTok{method =} \StringTok{"euler"}\NormalTok{)}
\NormalTok{pop\_size\_2 }\OtherTok{\textless{}{-}} \FunctionTok{ode}\NormalTok{(}\AttributeTok{func =}\NormalTok{ exponential\_model, }\AttributeTok{times =}\NormalTok{ times\_2, }\AttributeTok{y =}\NormalTok{ state, }\AttributeTok{parms =}\NormalTok{ parms, }\AttributeTok{method =} \StringTok{"euler"}\NormalTok{)}
\NormalTok{pop\_size\_3 }\OtherTok{\textless{}{-}} \FunctionTok{ode}\NormalTok{(}\AttributeTok{func =}\NormalTok{ exponential\_model, }\AttributeTok{times =}\NormalTok{ times\_3, }\AttributeTok{y =}\NormalTok{ state, }\AttributeTok{parms =}\NormalTok{ parms, }\AttributeTok{method =} \StringTok{"euler"}\NormalTok{)}

\CommentTok{\# Compare different time intervals}
\FunctionTok{par}\NormalTok{(}\AttributeTok{mfrow =} \FunctionTok{c}\NormalTok{(}\DecValTok{1}\NormalTok{,}\DecValTok{3}\NormalTok{))}
\FunctionTok{plot}\NormalTok{(N }\SpecialCharTok{\textasciitilde{}}\NormalTok{ time, }\AttributeTok{data =}\NormalTok{ pop\_size\_1, }\AttributeTok{main =} \StringTok{"Time intervals = 1"}\NormalTok{)}
\FunctionTok{curve}\NormalTok{(state[}\DecValTok{1}\NormalTok{]}\SpecialCharTok{*}\FunctionTok{exp}\NormalTok{(parms[}\DecValTok{1}\NormalTok{]}\SpecialCharTok{*}\NormalTok{x), }\AttributeTok{col =} \StringTok{"red"}\NormalTok{, }\AttributeTok{add =}\NormalTok{ T) }\CommentTok{\# correct curve}
\FunctionTok{plot}\NormalTok{(N }\SpecialCharTok{\textasciitilde{}}\NormalTok{ time, }\AttributeTok{data =}\NormalTok{ pop\_size\_2, }\AttributeTok{main =} \StringTok{"Time intervals = 0.1"}\NormalTok{)}
\FunctionTok{curve}\NormalTok{(state[}\DecValTok{1}\NormalTok{]}\SpecialCharTok{*}\FunctionTok{exp}\NormalTok{(parms[}\DecValTok{1}\NormalTok{]}\SpecialCharTok{*}\NormalTok{x), }\AttributeTok{col =} \StringTok{"red"}\NormalTok{, }\AttributeTok{add =}\NormalTok{ T) }\CommentTok{\# correct curve}
\FunctionTok{plot}\NormalTok{(N }\SpecialCharTok{\textasciitilde{}}\NormalTok{ time, }\AttributeTok{data =}\NormalTok{ pop\_size\_3, }\AttributeTok{main =} \StringTok{"Time intervals = 0.01"}\NormalTok{)}
\FunctionTok{curve}\NormalTok{(state[}\DecValTok{1}\NormalTok{]}\SpecialCharTok{*}\FunctionTok{exp}\NormalTok{(parms[}\DecValTok{1}\NormalTok{]}\SpecialCharTok{*}\NormalTok{x), }\AttributeTok{col =} \StringTok{"red"}\NormalTok{, }\AttributeTok{add =}\NormalTok{ T) }\CommentTok{\# correct curve}
\end{Highlighting}
\end{Shaded}

\includegraphics{bookdown-demo_files/figure-latex/unnamed-chunk-7-2.pdf}

\textbf{Part 3 - Solving exponential growth model with fluctuating growth rate}
Consider the model
\[
\frac{dN}{dt} = r(t)N \ \text{, } r(t) = \overline{r} + \sigma\sin(\omega t)
\]
where \(\overline{r}\) and \(\omega\) are constants.
The analytic solution of the ode model is
\[
N(t) = N_0\exp\{\overline{r}t - \frac{\sigma}{\omega}[\cos(\omega t) - 1]\}
\]

\begin{Shaded}
\begin{Highlighting}[]
\DocumentationTok{\#\#\#\#\#\# part 3 \#\#\#\#\#\#}
\DocumentationTok{\#\#\# Model specification}
\NormalTok{exponential\_model\_fluc }\OtherTok{\textless{}{-}} \ControlFlowTok{function}\NormalTok{(times, state, parms) \{}
  \FunctionTok{with}\NormalTok{(}\FunctionTok{as.list}\NormalTok{(}\FunctionTok{c}\NormalTok{(state, parms)), \{}
\NormalTok{    dN\_dt }\OtherTok{=}\NormalTok{ (r\_bar }\SpecialCharTok{+}\NormalTok{ sigma}\SpecialCharTok{*}\FunctionTok{sin}\NormalTok{(omega}\SpecialCharTok{*}\NormalTok{times))}\SpecialCharTok{*}\NormalTok{N  }\CommentTok{\# exponential growth equation}
    \FunctionTok{return}\NormalTok{(}\FunctionTok{list}\NormalTok{(}\FunctionTok{c}\NormalTok{(dN\_dt)))  }\CommentTok{\# return the results}
\NormalTok{  \})}
\NormalTok{\}}
\end{Highlighting}
\end{Shaded}

\begin{Shaded}
\begin{Highlighting}[]
\DocumentationTok{\#\#\# Parameters}
\NormalTok{times }\OtherTok{\textless{}{-}} \FunctionTok{seq}\NormalTok{(}\DecValTok{0}\NormalTok{, }\DecValTok{10}\NormalTok{, }\AttributeTok{by =} \FloatTok{0.1}\NormalTok{)  }\CommentTok{\# time steps to integrate over}
\NormalTok{state }\OtherTok{\textless{}{-}} \FunctionTok{c}\NormalTok{(}\AttributeTok{N =} \DecValTok{10}\NormalTok{)  }\CommentTok{\# initial population size}
\NormalTok{parms }\OtherTok{\textless{}{-}} \FunctionTok{c}\NormalTok{(}\AttributeTok{r\_bar =} \FloatTok{1.5}\NormalTok{, }\AttributeTok{sigma =} \DecValTok{5}\NormalTok{, }\AttributeTok{omega =} \DecValTok{2}\SpecialCharTok{*}\NormalTok{pi)  }\CommentTok{\# intrinsic growth rate}
\end{Highlighting}
\end{Shaded}

Plot \(r(t)\)

\begin{Shaded}
\begin{Highlighting}[]
\DocumentationTok{\#\#\# Fluctuating growth rate}
\NormalTok{r }\OtherTok{=}\NormalTok{ parms[}\DecValTok{1}\NormalTok{] }\SpecialCharTok{+}\NormalTok{ parms[}\DecValTok{2}\NormalTok{]}\SpecialCharTok{*}\FunctionTok{sin}\NormalTok{(parms[}\DecValTok{3}\NormalTok{]}\SpecialCharTok{*}\NormalTok{times)}
\FunctionTok{plot}\NormalTok{(r }\SpecialCharTok{\textasciitilde{}}\NormalTok{ times, }\AttributeTok{type =} \StringTok{"l"}\NormalTok{)}
\end{Highlighting}
\end{Shaded}

\includegraphics{bookdown-demo_files/figure-latex/unnamed-chunk-10-1.pdf}

\begin{Shaded}
\begin{Highlighting}[]
\DocumentationTok{\#\#\# Solving model}
\NormalTok{pop\_size }\OtherTok{\textless{}{-}} \FunctionTok{ode}\NormalTok{(}\AttributeTok{func =}\NormalTok{ exponential\_model\_fluc, }\AttributeTok{times =}\NormalTok{ times, }\AttributeTok{y =}\NormalTok{ state, }\AttributeTok{parms =}\NormalTok{ parms)}

\DocumentationTok{\#\#\# Plotting}
\FunctionTok{plot}\NormalTok{(N }\SpecialCharTok{\textasciitilde{}}\NormalTok{ times, }\AttributeTok{data =}\NormalTok{ pop\_size)}
\FunctionTok{curve}\NormalTok{(state[}\DecValTok{1}\NormalTok{]}\SpecialCharTok{*}\FunctionTok{exp}\NormalTok{(parms[}\DecValTok{1}\NormalTok{]}\SpecialCharTok{*}\NormalTok{x }\SpecialCharTok{{-}}\NormalTok{ parms[}\DecValTok{2}\NormalTok{]}\SpecialCharTok{/}\NormalTok{parms[}\DecValTok{3}\NormalTok{]}\SpecialCharTok{*}\NormalTok{(}\FunctionTok{cos}\NormalTok{(parms[}\DecValTok{3}\NormalTok{]}\SpecialCharTok{*}\NormalTok{x) }\SpecialCharTok{{-}} \DecValTok{1}\NormalTok{)), }\AttributeTok{add =}\NormalTok{ T, }\AttributeTok{col =} \StringTok{"red"}\NormalTok{) }\CommentTok{\# correct curve}
\end{Highlighting}
\end{Shaded}

\includegraphics{bookdown-demo_files/figure-latex/unnamed-chunk-11-1.pdf}

\begin{Shaded}
\begin{Highlighting}[]
\FunctionTok{plot}\NormalTok{(N }\SpecialCharTok{\textasciitilde{}}\NormalTok{ times, }\AttributeTok{data =}\NormalTok{ pop\_size, }\AttributeTok{log =} \StringTok{"y"}\NormalTok{)}
\FunctionTok{curve}\NormalTok{(state[}\DecValTok{1}\NormalTok{]}\SpecialCharTok{*}\FunctionTok{exp}\NormalTok{(parms[}\DecValTok{1}\NormalTok{]}\SpecialCharTok{*}\NormalTok{x }\SpecialCharTok{{-}}\NormalTok{ parms[}\DecValTok{2}\NormalTok{]}\SpecialCharTok{/}\NormalTok{parms[}\DecValTok{3}\NormalTok{]}\SpecialCharTok{*}\NormalTok{(}\FunctionTok{cos}\NormalTok{(parms[}\DecValTok{3}\NormalTok{]}\SpecialCharTok{*}\NormalTok{x) }\SpecialCharTok{{-}} \DecValTok{1}\NormalTok{)), }\AttributeTok{add =}\NormalTok{ T, }\AttributeTok{col =} \StringTok{"red"}\NormalTok{) }\CommentTok{\# correct curve}
\end{Highlighting}
\end{Shaded}

\includegraphics{bookdown-demo_files/figure-latex/unnamed-chunk-11-2.pdf}

Adjust \(\overline{r}\)

\begin{Shaded}
\begin{Highlighting}[]
\DocumentationTok{\#\#\# Parameters}
\NormalTok{times }\OtherTok{\textless{}{-}} \FunctionTok{seq}\NormalTok{(}\DecValTok{0}\NormalTok{, }\DecValTok{10}\NormalTok{, }\AttributeTok{by =} \FloatTok{0.1}\NormalTok{)  }\CommentTok{\# time steps to integrate over}
\NormalTok{state }\OtherTok{\textless{}{-}} \FunctionTok{c}\NormalTok{(}\AttributeTok{N =} \DecValTok{10}\NormalTok{)  }\CommentTok{\# initial population size}
\NormalTok{parms }\OtherTok{\textless{}{-}} \FunctionTok{c}\NormalTok{(}\AttributeTok{r\_bar =} \FloatTok{0.1}\NormalTok{, }\AttributeTok{sigma =} \DecValTok{5}\NormalTok{, }\AttributeTok{omega =} \DecValTok{2}\SpecialCharTok{*}\NormalTok{pi)  }\CommentTok{\# intrinsic growth rate}

\DocumentationTok{\#\#\# Fluctuating growth rate}
\NormalTok{r }\OtherTok{=}\NormalTok{ parms[}\DecValTok{1}\NormalTok{] }\SpecialCharTok{+}\NormalTok{ parms[}\DecValTok{2}\NormalTok{]}\SpecialCharTok{*}\FunctionTok{sin}\NormalTok{(parms[}\DecValTok{3}\NormalTok{]}\SpecialCharTok{*}\NormalTok{times)}
\FunctionTok{plot}\NormalTok{(r }\SpecialCharTok{\textasciitilde{}}\NormalTok{ times, }\AttributeTok{type =} \StringTok{"l"}\NormalTok{)}
\end{Highlighting}
\end{Shaded}

\includegraphics{bookdown-demo_files/figure-latex/unnamed-chunk-12-1.pdf}

\begin{Shaded}
\begin{Highlighting}[]
\DocumentationTok{\#\#\# Solving model}
\NormalTok{pop\_size }\OtherTok{\textless{}{-}} \FunctionTok{ode}\NormalTok{(}\AttributeTok{func =}\NormalTok{ exponential\_model\_fluc, }\AttributeTok{times =}\NormalTok{ times, }\AttributeTok{y =}\NormalTok{ state, }\AttributeTok{parms =}\NormalTok{ parms)}

\DocumentationTok{\#\#\# Plotting}
\FunctionTok{plot}\NormalTok{(N }\SpecialCharTok{\textasciitilde{}}\NormalTok{ times, }\AttributeTok{data =}\NormalTok{ pop\_size)}
\FunctionTok{curve}\NormalTok{(state[}\DecValTok{1}\NormalTok{]}\SpecialCharTok{*}\FunctionTok{exp}\NormalTok{(parms[}\DecValTok{1}\NormalTok{]}\SpecialCharTok{*}\NormalTok{x }\SpecialCharTok{{-}}\NormalTok{ parms[}\DecValTok{2}\NormalTok{]}\SpecialCharTok{/}\NormalTok{parms[}\DecValTok{3}\NormalTok{]}\SpecialCharTok{*}\NormalTok{(}\FunctionTok{cos}\NormalTok{(parms[}\DecValTok{3}\NormalTok{]}\SpecialCharTok{*}\NormalTok{x) }\SpecialCharTok{{-}} \DecValTok{1}\NormalTok{)), }\AttributeTok{add =}\NormalTok{ T, }\AttributeTok{col =} \StringTok{"red"}\NormalTok{) }\CommentTok{\# correct curve}
\end{Highlighting}
\end{Shaded}

\includegraphics{bookdown-demo_files/figure-latex/unnamed-chunk-12-2.pdf}

\chapter*{Week 3 - Logistic population growth and stability analysis}\label{week-3---logistic-population-growth-and-stability-analysis}
\addcontentsline{toc}{chapter}{Week 3 - Logistic population growth and stability analysis}

Credit to \href{https://genchanghsu.github.io/index.html}{Gen-Chang Hsu}

\textbf{Extra materials - Modeling discrete logistic models using for loops}

Model:
\[
N_{t+1} = N_t(1+r(1-\frac{N_t}{K}))
\]

\begin{Shaded}
\begin{Highlighting}[]
\DocumentationTok{\#\#\# (1) Define the discrete logistic growth equation}
\NormalTok{log\_fun }\OtherTok{\textless{}{-}} \ControlFlowTok{function}\NormalTok{(r, N, K)\{N }\SpecialCharTok{+}\NormalTok{ r}\SpecialCharTok{*}\NormalTok{N}\SpecialCharTok{*}\NormalTok{(}\DecValTok{1}\SpecialCharTok{{-}}\NormalTok{N}\SpecialCharTok{/}\NormalTok{K)\}}
\end{Highlighting}
\end{Shaded}

You may modify \(r\) to see the change in stability of equilibrium \(K\).

\begin{Shaded}
\begin{Highlighting}[]
\DocumentationTok{\#\#\# (2) Set the parameters}
\NormalTok{r }\OtherTok{\textless{}{-}} \FloatTok{1.8}
\NormalTok{K }\OtherTok{\textless{}{-}} \DecValTok{500}
\NormalTok{N0 }\OtherTok{\textless{}{-}} \DecValTok{10}
\NormalTok{time }\OtherTok{\textless{}{-}} \DecValTok{100}

\DocumentationTok{\#\#\# (3) Use for loop to iterate over the time sequence}
\NormalTok{pop\_size }\OtherTok{\textless{}{-}} \FunctionTok{data.frame}\NormalTok{(}\AttributeTok{times =} \DecValTok{1}\SpecialCharTok{:}\NormalTok{time)}
\NormalTok{pop\_size}\SpecialCharTok{$}\NormalTok{N[}\DecValTok{1}\NormalTok{] }\OtherTok{\textless{}{-}}\NormalTok{ N0}
\FunctionTok{head}\NormalTok{(pop\_size)}
\end{Highlighting}
\end{Shaded}

\begin{verbatim}
##   times  N
## 1     1 10
## 2     2 10
## 3     3 10
## 4     4 10
## 5     5 10
## 6     6 10
\end{verbatim}

\begin{Shaded}
\begin{Highlighting}[]
\ControlFlowTok{for}\NormalTok{(i }\ControlFlowTok{in} \DecValTok{2}\SpecialCharTok{:}\NormalTok{time)\{}
\NormalTok{  pop\_size}\SpecialCharTok{$}\NormalTok{N[i] }\OtherTok{\textless{}{-}} \FunctionTok{log\_fun}\NormalTok{(}\AttributeTok{r =}\NormalTok{ r, }\AttributeTok{N =}\NormalTok{ pop\_size}\SpecialCharTok{$}\NormalTok{N[i }\SpecialCharTok{{-}} \DecValTok{1}\NormalTok{], }\AttributeTok{K =}\NormalTok{ K)}
\NormalTok{\}}

\FunctionTok{head}\NormalTok{(pop\_size)}
\end{Highlighting}
\end{Shaded}

\begin{verbatim}
##   times         N
## 1     1  10.00000
## 2     2  27.64000
## 3     3  74.64171
## 4     4 188.93980
## 5     5 400.51775
## 6     6 543.95762
\end{verbatim}

\begin{Shaded}
\begin{Highlighting}[]
\DocumentationTok{\#\#\# (4) Population trajectory}
\FunctionTok{plot}\NormalTok{(N }\SpecialCharTok{\textasciitilde{}}\NormalTok{ times, }\AttributeTok{data =}\NormalTok{ pop\_size, }\AttributeTok{type =} \StringTok{"l"}\NormalTok{)}
\FunctionTok{abline}\NormalTok{(}\AttributeTok{h =}\NormalTok{ K, }\AttributeTok{col =} \StringTok{"red"}\NormalTok{)}
\FunctionTok{points}\NormalTok{(N }\SpecialCharTok{\textasciitilde{}}\NormalTok{ times, }\AttributeTok{data =}\NormalTok{ pop\_size)}
\end{Highlighting}
\end{Shaded}

\includegraphics{bookdown-demo_files/figure-latex/unnamed-chunk-15-1.pdf}

Here is a shiny app for the discrete logistic growth model.

Credit to \href{https://genchanghsu.github.io/index.html}{Gen-Chang Hsu}

  \bibliography{book.bib,packages.bib}

\end{document}
